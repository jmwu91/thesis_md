% !TeX root = ../main.tex

\providecommand{\tightlist}{%
  \setlength{\itemsep}{0pt}\setlength{\parskip}{0pt}
}

\chapter{資料分析}\label{ux8cc7ux6599ux5206ux6790}

\section{主成分分析(PCA)}\label{ux4e3bux6210ux5206ux5206ux6790pca}

本節透過主成分分析(Principal Component Analysis, PCA)對 30
餘個旅次行為變數進行降維,萃取可代表出行行為差異的主要構面,後續將以其作為使用者分群依據。

\subsection{Factor loading}\label{factor-loading}

\begin{longtable}[]{@{}
  >{\raggedright\arraybackslash}p{(\linewidth - 10\tabcolsep) * \real{0.3919}}
  >{\centering\arraybackslash}p{(\linewidth - 10\tabcolsep) * \real{0.1216}}
  >{\centering\arraybackslash}p{(\linewidth - 10\tabcolsep) * \real{0.1216}}
  >{\centering\arraybackslash}p{(\linewidth - 10\tabcolsep) * \real{0.1216}}
  >{\centering\arraybackslash}p{(\linewidth - 10\tabcolsep) * \real{0.1216}}
  >{\centering\arraybackslash}p{(\linewidth - 10\tabcolsep) * \real{0.1216}}@{}}
\toprule\noalign{}
\begin{minipage}[b]{\linewidth}\raggedright
Feature
\end{minipage} & \begin{minipage}[b]{\linewidth}\centering
PC1
\end{minipage} & \begin{minipage}[b]{\linewidth}\centering
PC2
\end{minipage} & \begin{minipage}[b]{\linewidth}\centering
PC3
\end{minipage} & \begin{minipage}[b]{\linewidth}\centering
PC4
\end{minipage} & \begin{minipage}[b]{\linewidth}\centering
PC5
\end{minipage} \\
\midrule\noalign{}
\endhead
\bottomrule\noalign{}
\endlastfoot
travel\_days & 0.074848 & 0.282796 & -0.119480 & -0.278332 & 0.024753 \\
total\_trips & 0.071285 & 0.296926 & -0.150977 & -0.268482 &
-0.095833 \\
avg\_trips\_per\_day & 0.027357 & 0.141389 & -0.147019 & -0.052192 &
-0.414660 \\
symmetrical\_days & -0.027288 & 0.293980 & -0.183776 & -0.228639 &
-0.163773 \\
symmetry\_ratio & -0.027732 & 0.159816 & -0.159616 & -0.115679 &
-0.247469 \\
avg\_travel\_time & 0.000264 & 0.162018 & -0.290944 & 0.335231 &
0.168572 \\
std\_travel\_time & 0.227504 & -0.025653 & -0.149048 & 0.198276 &
-0.136633 \\
weekday\_trip\_ratio & -0.044541 & 0.168333 & -0.035165 & -0.225991 &
0.441857 \\
weekend\_trip\_ratio & 0.044541 & -0.168332 & 0.035157 & 0.225989 &
-0.441855 \\
peak\_hour\_ratio & -0.053868 & 0.061551 & -0.005550 & -0.183756 &
0.114912 \\
total\_distance & 0.035383 & 0.297475 & -0.273886 & 0.035607 &
0.038432 \\
avg\_distance & -0.003092 & 0.165678 & -0.289710 & 0.345842 &
0.168331 \\
max\_distance & 0.135675 & 0.133928 & -0.280228 & 0.306150 & 0.052734 \\
min\_distance & -0.152118 & 0.084086 & -0.124452 & 0.261218 &
0.247421 \\
std\_distance & 0.218126 & -0.019389 & -0.158374 & 0.213428 &
-0.126825 \\
first\_boarding\_entropy & 0.288537 & -0.077371 & -0.022348 & -0.059241
& 0.012784 \\
last\_boarding\_entropy & 0.278198 & 0.034244 & -0.069781 & -0.154541 &
0.025303 \\
avg\_boarding\_entropy & 0.306537 & -0.024388 & -0.049358 & -0.114682 &
0.020469 \\
first\_boarding\_place\_entropy & 0.225090 & -0.164092 & 0.047021 &
0.045524 & 0.183025 \\
first\_alighting\_place\_entropy & 0.260636 & -0.156758 & -0.006790 &
0.049579 & 0.012067 \\
last\_boarding\_place\_entropy & 0.268824 & -0.102454 & -0.020367 &
0.017393 & -0.003434 \\
last\_alighting\_place\_entropy & 0.209180 & -0.125000 & 0.030430 &
-0.008440 & 0.225620 \\
unique\_first\_boarding\_place & 0.256534 & -0.068382 & 0.025942 &
-0.052357 & 0.149767 \\
unique\_first\_alighting\_place & 0.289524 & -0.022920 & -0.055341 &
-0.084783 & -0.051393 \\
unique\_last\_boarding\_place & 0.285742 & 0.049618 & -0.069275 &
-0.112814 & -0.070003 \\
unique\_last\_alighting\_place & 0.234281 & -0.024910 & -0.003030 &
-0.106229 & 0.181134 \\
total\_transfer & 0.107915 & 0.281155 & 0.308880 & 0.120038 &
-0.002115 \\
bus\_to\_mrt & 0.103389 & 0.258713 & 0.293466 & 0.104329 & 0.000615 \\
mrt\_to\_bus & 0.091850 & 0.280267 & 0.271741 & 0.112220 & -0.005077 \\
weekday\_transfer & 0.096391 & 0.283460 & 0.301454 & 0.110430 &
0.022430 \\
weekend\_transfer & 0.115217 & 0.113141 & 0.181912 & 0.109590 &
-0.142955 \\
avg\_transfer & 0.096428 & 0.244692 & 0.308819 & 0.156826 & 0.002471 \\
\end{longtable}

\subsection{使用者旅行行為構面}\label{ux4f7fux7528ux8005ux65c5ux884cux884cux70baux69cbux9762}

本研究透過主成分分析(Principal Component Analysis,
PCA)對使用者的旅次行為變數進行降維,以萃取出能代表公共運輸使用者旅次特性差異之核心構面,並進一步用於使用者集群分析。根據分析結果,前五個主成分(PC1
至
PC5)共解釋了約70\%的資料變異,其各自所代表之行為構面與內涵解釋詳述如下。

PC1 -
出行規律性(24.1\%)主要捕捉使用者在出行時間與地點上的規律性,代表使用者是否具有規律且可預測的出行行為。具體而言,此主成分的關鍵變數包括平均的上車時間熵、首趟行程的下車地點數、首趟行程的上車時間熵、末趟行程的上車地點數,以及末趟行程的上車時間熵。上述特徵皆直接反映使用者在時間與空間兩個維度的規律程度,熵值愈低則規律性愈高,顯示使用者的通勤行為愈固定;相反,熵值高者表示其出行的時間和地點相對較為隨機且變動。

PC2 -
使用強度與對稱性(18\%)則以使用強度與行程對稱性為核心,表達使用者搭乘公共運輸系統的整體強度及其行程規律性。此構面的重要變數包括總搭乘距離、總使用次數、對稱行程天數、工作日轉乘次數及總使用天數。上述變數反映使用者整體對公共運輸依賴的程度以及出行模式的穩定性,對稱行程天數尤其能夠揭示固定通勤行程的規律性。

PC3 -
轉乘行為(12.9\%)側重於轉乘行為,描繪使用者對於多模式或多路線公共運輸系統的利用狀況。主要變數涵蓋總轉乘次數、平均轉乘次數、工作日轉乘次數、公車轉捷運次數及平均旅行時間。此主成分顯示使用者對公共運輸網絡的整合性利用程度,轉乘行為越頻繁者通常表示對公共運輸系統的綜合使用更廣泛。

PC4 -
搭乘時間與距離(10.2\%)涵蓋使用者在旅次距離與時間的平均特性,反映出個體公共運輸旅程的典型規模與時空範圍。主要特徵包含平均搭乘距離、平均旅行時間、最大搭乘距離、總使用天數與總使用次數。此維度顯示使用者旅次距離與時間的特徵,進一步區分出偏好長距離或短距離旅行之不同族群。

PC5 -
平假日使用情形(6.3\%)則著重於使用者平日與假日的旅次分布特性,反映出使用者在不同類型日子的出行行為差異。核心特徵包括工作日旅次比率、假日旅次比率、平均每日使用次數、對稱行程比例與最小搭乘距離。此主成分能夠凸顯使用者是否有明顯的平假日差異,進一步辨識出偏好於工作日使用之通勤族群或偏好假日使用的休閒族群。

綜上所述,透過主成分分析萃取之五大主成分,清楚地顯示MaaS使用者於公共運輸旅次行為在規律性、使用強度與對稱性、轉乘特性、旅次長度及平假日使用差異等重要維度,有助於進一步輸入無監督機器學習模型進行更細緻之使用者族群特性分群。

\section{使用者分群結果}\label{ux4f7fux7528ux8005ux5206ux7fa4ux7d50ux679c}

\subsection{各群集中位數特性描述(旅次行為變數)}\label{ux5404ux7fa4ux96c6ux4e2dux4f4dux6578ux7279ux6027ux63cfux8ff0ux65c5ux6b21ux884cux70baux8b8aux6578}

\subsection{群集特性解釋與分類(通勤
vs.~非通勤)}\label{ux7fa4ux96c6ux7279ux6027ux89e3ux91cbux8207ux5206ux985eux901aux52e4-vs.-ux975eux901aux52e4}

\subsubsection{通勤族群 (C3 \textasciitilde{}
C6)}\label{ux901aux52e4ux65cfux7fa4-c3-c6}

\subsubsection{非通勤族群 (C0, C1, C2,
C7)}\label{ux975eux901aux52e4ux65cfux7fa4-c0-c1-c2-c7}

\subsection{效果量檢定}\label{ux6548ux679cux91cfux6aa2ux5b9a}

\begin{itemize}
\tightlist
\item
  (Kruskal-Wallis、Cliff's delta)
\end{itemize}

\subsection{群集解釋與摘要討論(整合行為特徵 +
結論)}\label{ux7fa4ux96c6ux89e3ux91cbux8207ux6458ux8981ux8a0eux8ad6ux6574ux5408ux884cux70baux7279ux5fb5-ux7d50ux8ad6}

\section{各群集之空間與個人特徵分布}\label{ux5404ux7fa4ux96c6ux4e4bux7a7aux9593ux8207ux500bux4ebaux7279ux5fb5ux5206ux5e03}

為進一步探討群集差異是否與個體屬性(如卡別)或站點周邊環境特性有關,本節彙整各群在捷運站點分布、使用者身分及運具使用比例之實際情形,以提供後續模型分析之基礎觀察。

\subsection{捷運站點分布}\label{ux6377ux904bux7ad9ux9edeux5206ux5e03}

\subsection{卡別比例}\label{ux5361ux5225ux6bd4ux4f8b}

\subsection{運具使用比例}\label{ux904bux5177ux4f7fux7528ux6bd4ux4f8b}

基於此分類結果,本研究為進一步探討在通勤族群內部是否存在不同的通勤行為機制,本研究聚焦於通勤族群,即C3
\textasciitilde{} C6
群集,欲了解影響其群集屬性的潛在因素。考量公共運輸使用行為容易受到站點周邊設施與接駁資源影響,遂選取捷運站周邊之公共運輸接駁系統(如公車與共享自行車站點數)及興趣點(POI,包括學校、醫院、商場等)作為解釋變數,並以群集代碼為應變項,建立多項羅吉特模式(Multinomial
Logit Model,
MNL),以探索各項空間與設施特徵是否為影響使用者通勤行為差異的重要因子。

\section{通勤族群多項羅吉特模式分析(MNL)}\label{ux901aux52e4ux65cfux7fa4ux591aux9805ux7f85ux5409ux7279ux6a21ux5f0fux5206ux6790mnl}

\%\%
接續前述分群與空間特徵結果,為驗證站點周邊環境是否影響通勤族群之歸屬,本節針對通勤導向之群集(Cluster
3 至 Cluster 6)建立多項羅吉特模式(Multinomial Logit Model,
MNL),以群集代碼為應變項,探討站點周邊公車與自行車接駁設施密度、各類興趣點數量、卡別等特徵是否為造成群集分化之潛在因子。
\%\%

\begin{quote}
僅針對 C3 \textasciitilde{} C6,將 4.3 的描述性結果交由 MNL
驗證其顯著性與影響力
\end{quote}

\section{結果與討論}\label{ux7d50ux679cux8207ux8a0eux8ad6}
