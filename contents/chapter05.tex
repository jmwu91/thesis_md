% !TeX root = ../main.tex

\providecommand{\tightlist}{%
  \setlength{\itemsep}{0pt}\setlength{\parskip}{0pt}
}

\chapter{結論與建議}\label{ux7d50ux8ad6ux8207ux5efaux8b70}

作為本研究的結論,主要對研究的結果進行總結、
並提出本研究的發現以及後續對於研究及政策制定方面的建議,
於5.1章對本研究的結論進行說明、5.2說明研究貢獻與限制,
最後說明後續研究得以發展的方向以及基於本研究結果對後續政策方向提出的建議。

\section{結論}\label{ux7d50ux8ad6}

本研究透過高雄市MaaS系統 - MeNGo
使用者電子票證資料進行公共運輸使用特性之分群研究,透過特徵工程從票證資料中萃取出32個代表使用者公共運輸旅次特性變數,接著以主成分分析進行變數降維,發掘出五個不同構面的使用者行為樣態,接續將五個主成分作為分群演算法的輸入變數進行使用者分群,最終得到八群不同公共運輸使用特性之使用者,並依其公共運輸使用頻率將其區分為通勤及非通勤族,類型及命名如下表
xxx
所示,最後針對通勤族群輔以多項羅吉特模式針對旅次特性外變數進行各群集使用者其個人及使用站點特性偏好之探討,得出以下結論。

\begin{enumerate}
\def\labelenumi{\arabic{enumi}.}
\tightlist
\item
  於目前高雄 MaaS 系統 - MeNGo
  使用者中存在兩大公共運輸使用者族群,通勤及非通勤族群,於各群使用者間具有相異公共運輸使用特性,分別根據代表其通勤或休閒目的,並根據使用者之起訖點位置分布、賴以使用之公共運輸系統,會造成群集間相差甚異的出行模式。
\end{enumerate}

\begin{itemize}
\tightlist
\item
  通勤族群

  \begin{itemize}
  \tightlist
  \item
    長距離通勤族
  \item
    高轉乘通勤族
  \item
    短距離通勤族
  \item
    多目的通勤族
  \end{itemize}
\item
  非通勤族群

  \begin{itemize}
  \tightlist
  \item
    混合使用者
  \item
    間歇平日使用者
  \item
    間歇假日使用者
  \item
    一次性使用者
  \end{itemize}
\end{itemize}

根據上述不同族群,理解到在MaaS使用者中具有各自相異特性,並與現今MaaS使用者之相關研究指出,主要以公共運輸系統之組合使用者以及高度的時空規律性,與本研究實證結果相吻合。

透過時空規律性作為使用者公共運輸使用特徵,得以去理解各類使用者在對於系統使用的規律程度,通勤者因通勤通學之固定時間需求,造成其具有較高的時空規律作為其最主要的使用特性,亦反映在運具選擇上,規律通勤者多以固定時間及班距較小的捷運系統作為其主要使用運具。

\section{研究貢獻與限制}\label{ux7814ux7a76ux8ca2ux737bux8207ux9650ux5236}

具體而言,本研究之貢獻與限制如以下:

研究貢獻: 1.
針對多元旅運整合服務(MaaS)使用者之電子票證資料建構一套資料處理流程,用以處理缺失資料及使用者特徵萃取過程。
2.
萃取出32個代表不同公共運輸旅次特性的使用者特徵,並以主成分分析得到五大行為構面,分別為使用規律性、使用強度與對稱性、轉乘行為、搭乘時間與距離與平假日使用,以上五大使用者行為構面。
3. 針對高雄市MaaS系統 - MeNGo
作為實證分析標的,從中分出八個不同特性之使用者集群,作為後續相關單位改善服務之參考。

研究限制: 1.
因本研究取得資料在公車、臺鐵及共享自行車之票證刷卡中,無法取得該旅次之起訖位置,因而僅能透過捷運之刷卡資料搭配其他運具之使用情形進行分析。

\section{後續建議}\label{ux5f8cux7e8cux5efaux8b70}

對政策的建議: 1.
根據較為鮮明的使用者通勤特性,進行不同套票方案的設計,如長短距離、高轉乘及多目的使用者,去根據可使用運具組合、可搭乘距離、搭乘時間限制及套票天數期限等組合進行販售,以精準吸引不同類型之使用者,以成功吸引潛在使用者進行套票購買與使用。

\begin{enumerate}
\def\labelenumi{\arabic{enumi}.}
\setcounter{enumi}{1}
\item
  並根據非通勤族使用者於本研究所呈現之特性,本研究認為因高雄機車使用之高比例的特性,以及在捷運路網尚未完善的情況下,建議可以增加公車服務的時空涵蓋率,提高公車服務於公共運輸路網中的接駁能力,使其得以有效配合目前以捷運為主要公共運輸骨幹的特性,增加捷運站點對於使用者的可及性,藉以提升使用者透過公共運輸達成其旅次目的。
\item
  在資料取得上面的限制,建議於政府開放資料部分,得以去增加
\end{enumerate}

==做這些時間跟空間熵讓我可以去多看到一些甚麼不同的東西去說明現況。他們到底在實際資料上面有看到甚麼。多少人行程很不固定(ch4)==
