% !TeX root = ../main.tex

\providecommand{\tightlist}{%
  \setlength{\itemsep}{0pt}\setlength{\parskip}{0pt}
}

\chapter{緒論}\label{ux7dd2ux8ad6}

\section{研究背景}\label{ux7814ux7a76ux80ccux666f}

多元整合旅運服務 (Mobility as a Service, MaaS)
是一種透過單一平台介面整合多種運輸系統(包含公共運輸系統、共享運具、副大眾運輸等),提供使用者無縫運輸服務的系統模式,並結合行程規劃與支付系統等服務,以滿足使用者多樣化的出行需求
(Wong et al., 2020; Matowicki et al.,
2024)。透過此種整合方式,期望能有效降低私人運具(如小汽車及機車)之擁有及使用,並利用資訊系統技術提供即時、便捷與經濟之服務模式,以創造私人運具使用者轉移使用MaaS之利基。

MaaS
系統的核心內涵包含兩項主要服務特性:其一為多元運具之整合,提供跨運具、時空無縫的套裝運輸服務;其二為結合資通訊技術,透過智慧平台提供即時且有效的出行資訊及支付服務。此外,根據使用者需求,MaaS系統亦可區分為不同服務類型,例如以都會區內的通勤旅次為主要服務對象,並針對上班族、學生、高齡者或境外旅客等不同族群的特性,提供客製化及差異化的整合服務方案,提供使用者進行選擇。

公共運輸作為 MaaS 系統的核心服務項目,過往研究指出,多式聯運
(Multi-modal transportation) 的使用者屬於 MaaS
早期的潛在使用者,顯示公共運輸系統與 MaaS 之間的密切關聯 (Alonso et al.,
2020)。此外,MaaS的發展亦與永續運輸具有高度相關,其整合的運輸模式能夠有效促進低碳出行行為,提高公共運輸使用率,進而達成減少私人運具依賴的目標
(Alyavina et al., 2020)。

全球各地的實證研究亦顯示,MaaS
系統具備顯著提升公共運輸使用者出行滿意度的潛力 (Sochor et al.,
2016)。例如,瑞典的 UbiGo
試辦計畫顯示64\%的參與者減少私人運具的使用;奧地利的 Smile
計畫則有21\%的參與者降低私人運具的使用頻率 (Alonso-González et al.,
2020)。然而,雖然MaaS能夠影響使用者的運具選擇,但卻難以完全取代私人運具,部分使用者傾向同時使用MaaS與私人運具,顯示MaaS在某些情境下更像是一種私人運具之外的補充選擇,而非完全替代
(Alyavina et al., 2020)。

儘管如此,目前針對 MaaS
使用者實際公共運輸使用行為特性的實證研究仍相對有限,進一步的研究與分析成為本研究的動機。

\section{研究動機}\label{ux7814ux7a76ux52d5ux6a5f}

在 MaaS 的發展過程中,公共運輸通常做為主要骨幹,且早期的 MaaS
使用者也多為習慣多式聯運(Multi-modal
transportation)之使用者,其行為模式與公共運輸的特性密切相關 {[}Alonso
et al., 2020{]},說明在MaaS發展中公共運輸是為不可或缺的一部分。

本研究透過分析 MeNGo
使用者的電子票證資料,挖掘使用者的旅行行為模式,進一步理解不同族群的使用行為、活動地點特性。希望透過這樣的分析,找出影響
MeNGo 使用的關鍵因素及其客群特性,進而提供 MaaS
系統營運改善建議,並為高雄市的公共運輸政策提供實務參考,期以促進整體公共運輸環境的發展。

\section{研究目的}\label{ux7814ux7a76ux76eeux7684}

本研究的主要目的在於透過 MeNGo
的電子票證資料,分析使用者的特性,並探討其對 MaaS
系統與公共運輸規劃的影響,具體研究目的如下:

\begin{enumerate}
\def\labelenumi{\arabic{enumi}.}
\tightlist
\item
  探索 MeNGo
  使用者的旅行行為,分析出行規律與趨勢,了解使用者行為的對稱性與規律性。
\item
  透過無監督學習方法,找出旅行行為相似的使用者集群,以將不同類型的使用者分為不同集群。
\item
  分析各個集群的使用者個體特性與主要活動場域,探討這些因素是否影響他們的出行模式,並從中發掘
  MaaS 服務的改善方向。
\end{enumerate}

\section{研究架構}\label{ux7814ux7a76ux67b6ux69cb}
