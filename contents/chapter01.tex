% !TeX root = ../main.tex

\providecommand{\tightlist}{%
  \setlength{\itemsep}{0pt}\setlength{\parskip}{0pt}
}

\chapter{緒論}\label{ux7dd2ux8ad6}

\section{研究背景}\label{ux7814ux7a76ux80ccux666f}

交通行動服務 (Mobility as a Service,
MaaS)是一種整合多種運輸系統的服務模式,包括公共運輸、共享運具、計程車等,透過單一平台提供無縫的運輸服務,期望降低私人運具的使用需求
(Wong et al., 2020; Matowicki et al., 2024)。其中,公共運輸通常是 MaaS
系統的核心骨幹,而多式聯運 (Multi-modal transportation) 的使用者往往是
MaaS 早期的潛在客群,顯示出公共運輸與 MaaS 之間的密切關聯 (Alonso et
al., 2020)。此外,MaaS
的發展亦與永續運輸息息相關,其整合的運輸模式能夠有效促進低碳出行,提高公共運輸的使用率,進而達成減少私人運具依賴的目標
(Alyavina et al., 2020)。

在全球各地,MaaS
的發展已展現出其提升公共運輸吸引力的潛力。研究顯示,MaaS
能夠顯著提升公共運輸使用者的出行滿意度 (Sochor et al., 2016)。瑞典的
UbiGo 試辦計畫顯示,64\% 的參與者減少了私人運具的使用,而奧地利的 Smile
試辦計畫則顯示 21\% 的參與者降低了私人運具的依賴 (Alonso-González et
al., 2020)。然而,MaaS
雖然能夠影響使用者的運具選擇,卻難以完全取代私人運具。部分使用者仍然傾向同時使用
MaaS 與私人運具,顯示 MaaS
在某些情境下更像是一種私人運具的補充,而非完全的替代方案 (Alyavina et
al., 2020)。

\section{研究動機}\label{ux7814ux7a76ux52d5ux6a5f}

在 MaaS 的發展過程中,公共運輸通常做為主要骨幹,且早期的 MaaS
使用者也多為習慣多式聯運(Multi-modal
transportation)之使用者,其行為模式與公共運輸的特性密切相關 {[}Alonso
et al., 2020{]},說明在MaaS發展中公共運輸是為不可或缺的一部分。

本研究透過分析 MeNGo
使用者的電子票證資料,挖掘使用者的旅行行為模式,進一步理解不同族群的使用行為、活動地點特性。希望透過這樣的分析,找出影響
MeNGo 使用的關鍵因素及其客群特性,進而提供 MaaS
系統營運改善建議,並為高雄市的公共運輸政策提供實務參考,期以促進整體公共運輸環境的發展。

\section{研究目的}\label{ux7814ux7a76ux76eeux7684}

本研究的主要目的在於透過 MeNGo
的電子票證資料,分析使用者的特性,並探討其對 MaaS
系統與公共運輸規劃的影響,具體研究目的如下:

\begin{enumerate}
\def\labelenumi{\arabic{enumi}.}
\tightlist
\item
  探索 MeNGo
  使用者的旅行行為,分析出行規律與趨勢,了解使用者行為的對稱性與規律性。
\item
  透過無監督學習方法,找出旅行行為相似的使用者集群,以將不同類型的使用者分為不同集群。
\item
  分析各個集群的使用者個體特性與主要活動場域,探討這些因素是否影響他們的出行模式,並從中發掘
  MaaS 服務的改善方向。
\end{enumerate}

\section{研究架構}\label{ux7814ux7a76ux67b6ux69cb}
