% !TeX root = ../main.tex

\providecommand{\tightlist}{%
  \setlength{\itemsep}{0pt}\setlength{\parskip}{0pt}
}

\chapter{緒論}\label{ux7dd2ux8ad6}

\section{研究背景}\label{ux7814ux7a76ux80ccux666f}

多元整合旅運服務 (Mobility as a Service, MaaS)
是為透過單一平台介面整合多種運輸系統(包含公共運輸系統、共享運具、副大眾運輸等),且結合行程規劃與支付系統等日常出行所需,藉此讓使用者得以擁有無縫體驗之運輸服務,並滿足使用者多樣化的出行需求
(Wong et al., 2020; Matowicki et al.,
2024)。透過此種服務整合方式,提昇使用者透過公共運輸進行日常的出行需求之意願,期望能有效降低私人運具之使用與擁有,並利用愈加發達的資訊系統技術提供即時、便捷與準確之服務體驗,以創造私人運具使用者轉移使用公共運輸系統之利基。

MaaS
系統的核心包含兩項主要服務特性:其一為多元運具之整合,提供跨運具、時空無縫的套裝運輸服務;其二為結合資訊技術,透過平台提供即時且有效的出行所需資訊及支付服務。且基於服務面向之受眾特性、服務地區之不同,MaaS系統亦可區分為不同服務類型,例如以都會區內的通勤旅次為主要服務對象,並針對上班族、學生、高齡者或境外旅客等不同族群的特性,提供客製化及差異化的整合服務方案,提供使用者進行選擇;亦或以服務偏鄉使用者為主,主要服務於公共運輸服務量能不足之偏遠地區,透過平台介面預約派車等服務滿足使用者日常出行及接送等需求;以及於城際間的國道客運運輸服務等,由上可知,基於不同的使用者特性去發展出各項不同型態之MaaS服務,其目的皆為提供使用者一個完整的運輸服務體驗,進而提升對於公共運輸系統的使用意願。

目前於國內MaaS系統的發展,主要以都會型的MaaS系統 - MeNGo、城際型MaaS系統
- TBS 台北轉運站以及數個偏鄉型及觀旅型MaaS系統,如:台東 - TTGO、花蓮 -
Hualien
Yo真行,以上為目前國內正在營運之MaaS系統。其中以高雄MeNGo作為整合服務最完善之MaaS系統,其服務結合高雄市境內捷運、公車、輕軌、渡輪、共享運具及各類大眾及副大眾運輸系統,是為亞洲第一個MaaS服務示範區。且於2023年整合TPASS政策後,將其服務範圍擴展至台南及屏東,期望能透過MeNGo整合之公共運輸服務打造南高屏永續生活圈。

於過往研究指出,多式聯運 (Multi-modal transportation)
的公共運輸使用者屬於 MaaS 早期的潛在使用者,顯示公共運輸系統與 MaaS
之間的密切關聯(Alonso et al.,
2020),得以說明在MaaS系統的服務中,公共運輸系統的整合作為其核心服務。此外,MaaS的發展亦與永續運輸模式之發展具有高度相關,其整合各類不同運輸系統的服務方式得以有效提高公共運輸使用率,進而達成減少使用者對於私人運具的依賴,並達成環境永續的目標
(Alyavina et al., 2020)。

全球各地的實證研究亦顯示,MaaS
系統具備顯著提升公共運輸使用者出行滿意度的潛力 (Sochor et al.,
2016)。例如,瑞典的 UbiGo
試辦計畫顯示64\%的參與者減少私人運具的使用;奧地利的 Smile
計畫則有21\%的參與者降低私人運具的使用頻率 (Alonso-González et al.,
2020)。然而,雖然MaaS能夠影響使用者的運具選擇,但卻難以完全取代私人運具,部分使用者傾向同時使用MaaS與私人運具,說明MaaS在某些情境下更傾向為私人運具之外的替代選擇,並不足以成為使用者的主要出行方式
(Alyavina et al., 2020)。

儘管如此,目前針對 MaaS
使用者實際公共運輸使用行為特性的實證研究仍相對有限,進一步的研究與分析成為本研究的動機。

\section{研究動機}\label{ux7814ux7a76ux52d5ux6a5f}

在都會型 MaaS
系統的發展中,公共運輸通常做為主要骨幹,用以承載都市中通勤者日常的使用需求,且早期的
MaaS 使用者也多為習慣多式聯運(Multi-modal
transportation)之使用者,其行為模式與公共運輸的特性密切相關 {[}Alonso
et al., 2020{]},說明在MaaS發展中公共運輸是為不可或缺的一部分。

本研究透過分析都會型 MaaS
系統之使用者公共運輸電子票證資料,挖掘使用者的旅行行為模式,進一步理解其公共運輸使用行為不同之使用者及其活動地點特性。希望透過本研究,了解影響
MaaS
使用的關鍵因素及其客群特性,使MaaS營運單位得以對現有服務根據公共運輸使用特性進行方案拆解,設計出滿足多數公共運輸使用者之MaaS服務方案,並同時吸引潛在使用者,加深並為高雄市的公共運輸政策提供實務參考,期以促進整體公共運輸環境的發展。

\section{研究目的}\label{ux7814ux7a76ux76eeux7684}

本研究的主要目的在於透過 MaaS
系統中公共運輸使用者之電子票證資料,理解其使用者的特性,並探討其對 MaaS
系統與公共運輸規劃的影響,具體研究目的如下:

\begin{enumerate}
\def\labelenumi{\arabic{enumi}.}
\tightlist
\item
  探索 MeNGo
  使用者的旅行行為,分析出行規律與趨勢,了解使用者行為的對稱性與規律性。
\item
  透過無監督學習方法,找出旅行行為相似的使用者集群,以將不同類型的使用者分為不同集群。
\item
  分析各個集群的使用者個體特性與主要活動場域,探討這些因素是否影響他們的出行模式,並從中發掘
  MaaS 服務設計的改善方向。
\end{enumerate}

\section{研究架構}\label{ux7814ux7a76ux67b6ux69cb}
