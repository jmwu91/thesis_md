% !TeX root = ../main.tex

\providecommand{\tightlist}{%
  \setlength{\itemsep}{0pt}\setlength{\parskip}{0pt}
}

\chapter{文獻回顧}\label{ux6587ux737bux56deux9867}

\section{多元旅運整合服務 (Mobility as a Service,
MaaS)}\label{ux591aux5143ux65c5ux904bux6574ux5408ux670dux52d9-mobility-as-a-service-maas}

\subsection{MaaS
定義與整合概念}\label{maas-ux5b9aux7fa9ux8207ux6574ux5408ux6982ux5ff5}

Mobility as a
Service(MaaS)為一種整合多元交通工具於單一平台的運輸服務模式,透過行程規劃、訂票、支付等功能,提供使用者無縫、便利的多模式移動體驗(Jittrapirom
et al., 2017)。然而,Hensher et al.(2021)指出,即便 MaaS
討論已逾十年,其具體定義與架構仍未有一致共識。目前多數研究認為 MaaS
應具備服務整合、使用者導向、移動性打包(mobility
packaging)與數位化平台等特徵(Smith, 2020)。此外,MaaS
的整合層級可涵蓋交通工具(公共運輸、共享運具等)、服務功能(預訂、資訊、支付)以及參與機構(營運商、整合者、政府部門)等。此一模式強調從個人擁有交通工具轉向以使用服務為主,重塑個人移動行為與交通供給方式。

\subsection{使用者特性、潛在使用者}\label{ux4f7fux7528ux8005ux7279ux6027ux6f5bux5728ux4f7fux7528ux8005}

現有 MaaS
相關研究多以問卷調查資料為基礎,針對潛在使用意願、接受動機與個人特性進行分群與行為預測,從中辨識可作為
MaaS 導入對象的關鍵族群。

Alonso-González 等人(2020)針對荷蘭都市地區進行 MaaS
使用潛力的態度分群分析,藉由潛在類別群集分析(Latent Class Cluster
Analysis, LCCA)將樣本分為五類,包含高度傾向使用 MaaS 的「FLEXI-ready
individuals」、強烈偏好傳統大眾運輸的「Multimodal public transportation
supporters」,以及態度保守、抗拒創新科技的「Anti new-mobility
individuals」等。該研究涵蓋五大態度構面(行動整合、共享運具偏好、對共享運具之疑慮、手機應用程式接受度與付費意願),其中發現,潛在
MaaS
使用者多為年輕、高學歷、住在都會區且交通模式使用多樣化的族群。相對地,對
MaaS
接受度較低的群體常具備高度私人運具依賴與科技抗拒傾向。該研究強調,針對使用者類型進行套票方案差異化設計為
MaaS 成功推廣的關鍵因素。

Aguiar 等人(2023)則針對西班牙馬德里地區進行 K-means 分群,將潛在 MaaS
使用者歸為「MaaS
熱衷者」、「創新懷疑者」與「反科技者」三群。該研究整合個人生活態度、科技接受度與環保價值觀等潛在變項,發現對
MaaS
接受度較高者,多具備強烈環保意識、公共運輸與單車使用頻率高,並對新科技與多元運具保持開放態度。研究明確指出,MaaS
潛在使用者的社經條件、價值信念與通勤行為密切相關,尤其在性別、年齡、交通選擇多樣性與科技親和力等層面展現顯著差異。此結果進一步支持
MaaS 推廣策略應對族群差異加以回應。

Matyas(2020)採取質性研究途徑,透過倫敦地區 30 位受訪者的深入訪談,分析
MaaS 套餐在改變出行選擇與交通習慣上的潛力。研究發現使用者會將 MaaS
套餐中的運具分類為「必要模式」、「可考慮模式」與「排除模式」,其中,公共運輸為幾乎所有人歸為必要模式的項目,尤以通勤為主的日常行為最為關鍵。此外,MaaS
被視為可擴展個人「交通選項認知集合(perceived choice
set)」的策略,協助使用者將過往未考慮的出行方式納入行程選項。該研究亦指出,若欲促進使用者行為轉變,需針對其「可考慮模式」進行精準推廣,特別是在通勤高度受限的家庭與女性族群,通勤行為經常是交通決策的核心模式,也是
MaaS 設計與導入時的重要參照。

進一步地,Duan 等人(2022)運用類神經網路(Artificial Neural Network,
ANN)模型,針對澳洲地區進行 MaaS
使用意圖之預測,並針對不同出行目的(社交、日常、工作)分別建構模型。結果顯示,通勤目的之
MaaS
使用行為最具預測性,其主要受年齡、就業、子女數、旅行距離與時間、公共運輸票證持有等因素顯著影響。研究指出,通勤者相較於其他出行目的者,具備更穩定的行為模式與明確的需求特徵,為
MaaS
發展初期最具潛力的使用族群之一。此研究亦補足過往文獻對旅次目的差異之探討不足,顯示不同出行情境下的
MaaS 採用傾向需納入政策與系統設計的考量。

綜上所述,從潛在使用者及使用意願到以使用特性及目的進行的行為預測研究,MaaS
使用者特性研究已逐漸建構出一套清晰的輪廓:以通勤目的及年輕族群為主,並透過多運具使用的交通模式、科技親和力高者及環保價值觀者,是為MaaS推廣的主要受眾。下一節將進一步轉向以實際使用紀錄為基礎的研究,探討
MaaS
使用行為的時空特性與群體異質性,彌補僅從意願推論而未驗證實際行為的限制。

\subsection{MaaS使用者行為}\label{maasux4f7fux7528ux8005ux884cux70ba}
