% !TeX root = ../main.tex

\providecommand{\tightlist}{%
  \setlength{\itemsep}{0pt}\setlength{\parskip}{0pt}
}

\chapter{文獻回顧}\label{ux6587ux737bux56deux9867}

本章旨在系統性回顧並評析多元旅運整合服務(Mobility as a Service,
MaaS)與公共運輸使用者行為分析之相關研究,以定位現有知識的核心貢獻與仍待深入之議題。首先,將說明
MaaS 的定義、概念演變與各國試辦計畫之實證成果,勾勒出 MaaS
研究與實務推動的現況與缺口。其次,聚焦於使用者層面,彙整關於潛在採用者、使用者特性與實際使用行為的研究成果,藉此描繪不同族群於
MaaS
生態系中的角色與需求。最後,延伸至公共運輸使用者行為之文獻,透過電子票證資料與使用者旅運行為特徵等方向切入,了解可供
MaaS
研究借鏡的方法與觀察,期為後續研究方法探索與實務應用奠定完善且具體的基礎。

\section{多元旅運整合服務 (Mobility as a Service,
MaaS)}\label{ux591aux5143ux65c5ux904bux6574ux5408ux670dux52d9-mobility-as-a-service-maas}

\subsection{MaaS 定義與整合}\label{maas-ux5b9aux7fa9ux8207ux6574ux5408}

根據過往文獻指出(Hietanen, 2014;Jittrapirom et al., 2017;Kamargianni
et al., 2016;Wong, Hensher, \& Mulley, 2020),多元旅運整合服務
(Mobility as a Service, MaaS)
為一種創新的都市運輸整合模式,其服務核心強調
``透過單一數位平台整合多元交通運具,提供使用者從行程規劃、預訂到付款的運輸整合服務''
。此概念相較過往以「擁有運具」為主的個人交通方式,轉向以「按需使用」為導向的服務型交通模式,其核心理念在於運具整合與以使用者為中心,期望能提供無縫、便利且個人化的出行體驗(Giesecke,
Surakka, \& Hakonen, 2016;Pickford \& Chung, 2019)。MaaS
的實現依賴多項數位科技發展,包括智慧型手機、GPS、電子票證與支付系統、即時資訊處理技術等,這些技術構成
MaaS 平台運作的基礎條件(Nemtanu et al., 2016;He \& Csiszár, 2020)。

為區分不同程度的 MaaS 整合服務,Sochor et al.(2018)提出 MaaS
整合成熟度模型,將其分為五個層級:第0級為無整合,僅為單一運具服務;第1級為資訊整合,如
Google Maps、Moovit
提供多模式路線規劃;第2級整合預訂與支付功能,例如台灣的
T-Express;第3級則為服務整合,能提供月租型出行套票,如芬蘭 Whim 或高雄
MeN Go;最高的第 4
級則進一步將社會目標納入,例如鼓勵永續移動、改善空氣品質與促進社會公平。此分類架構不僅幫助辨識
MaaS 系統發展階段,也有助於政策制定者與服務提供者釐清各自角色與責任。

然而在實際推動過程中,全面整合型 MaaS
系統在制度、資源與商業模式上往往面臨高度挑戰,因此 Pickford 與
Chung(2019)提出「MaaS
Lite」概念,主張從較簡化的模式出發,先行整合地方性常見出行需求(如接駁巴士與捷運),以降低推動門檻與導入風險。整體而言,MaaS
不僅是一種技術或商業創新,更被視為實現永續都市交通的重要策略。藉由促進運具間的轉乘便利性與資源共享,MaaS
有望降低私人車輛依賴,提升運輸系統效率,進而達成減碳與改善都市移動品質之目標(Gould,
Wehrmeyer, \& Leach, 2015;Wong, Hensher, \& Mulley, 2020)。

於先前各國的MaaS實際試辦計畫中可以觀察到,MaaS
系統展現出了其降低私人運具使用的潛力,例如。芬蘭赫爾辛基的 Whim
營運資料顯示,該系統使用者中有 95.2 \% 的旅次依賴公共運輸使用, 3.8 \%
透過計程車與共享汽車達成旅次目的,且約有 68 \%
的旅次集中於可及性最高之公共運輸廊帶,證明 MaaS
作為強化公共運輸網路之有效手段,並非取代公共運輸;同時,Whim 使用者在 1
公里內首末段大量使用共享自行車與計程車,其中 97 \% 的單車旅次少於 30
分鐘、87 \% 之計程車使用旅次短於 5
公里,於芬蘭的案例中,著實呈現「以公共運輸為主、共享為輔」之多模式運輸型態
(Ramboll, 2019)。並於澳洲雪梨 MaaS
試辦計畫將公共運輸月票、巴士與火車折抵設定為所有MaaS套餐方案之基礎,並以計程車與共享汽車額度作為附加價值提供使用者運用,凸顯出「整合票證」在吸引早期使用者與落實永續目標上的關鍵性;其研究結果進一步證實,若使用者選擇訂閱方案的機率每增加
1 個百分點,月度私家車行駛距離平均減少約 67 公里 (Hensher et al.,
2021)。且於瑞典 UbiGo 與奧地利 Smile 等較早期的MaaS試辦計畫中,研究發現
44 \% 與 21 \%
的計畫參與者在六個月內降低私家車使用;研究指出,唯有達到資訊、票證、付款與客製化套票方案等因素的高度整合(Integration
Level 3--4),才能真正且有效地讓使用者的出行模式進行轉移 (Durand et al.,
2018)。

綜合而言,透過過去MaaS試辦計畫進行的實證研究,皆一致證明「公共運輸為
MaaS
骨幹、搭配多模式運輸系統及第一/最後一哩路之接駁與訂閱制的套票銷售模式」能有效減少使用者對私人運具的依賴,並促進城市間公共運輸系統的使用率及使用者的使用意願,並更能去體現出MaaS服務出現的核心概念,透過一個平台去完成日常所需的運輸服務。

\subsection{MaaS
使用者特性與潛在使用者}\label{maas-ux4f7fux7528ux8005ux7279ux6027ux8207ux6f5bux5728ux4f7fux7528ux8005}

於MaaS潛在使用者及使用者特性相關之研究,目前多以問卷調查搭配多種分析方法,如潛在類別分析、ANN預測模型等,以使用者意向資訊為基礎進行MaaS使用者之類別分析,主要研究羅列以下詳述。

\textbf{Alonso-González 等人(2020)} 針對荷蘭都市地區進行 MaaS
使用潛力的態度分群分析,藉由潛在類別群集分析(Latent Class Cluster
Analysis, LCCA)將樣本分為五類,包含高度傾向使用 MaaS 的「FLEXI-ready
individuals」、強烈偏好傳統大眾運輸的「Multimodal public transportation
supporters」,以及態度保守、抗拒科技的「Anti new-mobility
individuals」等使用者群體。該研究涵蓋五大態度構面,抱括運具整合、共享運具偏好、對共享運具之使用疑慮、手機應用程式接受度與付費意願,其中發現,潛在
MaaS
使用者多為年輕、高學歷、住在都會區且運輸模式使用多樣化的族群。相對地,對
MaaS
接受度較低的群體常高度依賴私人運具及科技抗拒傾向。該研究強調,針對使用者類型進行套票方案差異化設計為
MaaS 成功推廣的關鍵因素。

\textbf{Aguiar 等人(2023)} 則針對西班牙馬德里地區進行 K-means
分群,將潛在 MaaS 使用者歸為「MaaS
熱衷者」、「創新懷疑者」與「反科技者」三群。該研究透過整合個人生活態度、科技接受度與環保價值觀等潛在變量,發現對
MaaS
接受度較高者,多具備強烈環保意識、公共運輸與自行車使用頻率高,並對新科技與多元運具保持開放態度。研究明確指出,MaaS
潛在使用者的社經條件、價值信念與通勤行為密切相關,尤其在性別、年齡、交通選擇多樣性與科技親和力等層面展現顯著差異。此結果進一步支持
MaaS 推廣策略應對族群差異加以回應。

\textbf{Matyas(2020)} 採取質性研究途徑,透過倫敦地區 30
位受訪者的深入訪談,分析 MaaS
套餐在改變出行選擇與交通習慣上的潛力。研究發現使用者會將 MaaS
套餐中的運具分類為「必要模式」、「可考慮模式」與「排除模式」,其中,公共運輸為幾乎所有人歸為必要模式的項目,尤以通勤為主的日常行為最為關鍵。此外,MaaS
被視為可擴展個人交通選項認知集合(perceived choice
set)的策略,協助使用者將過往未考慮的出行方式納入行程選項。該研究亦指出,若欲促進使用者之出行行為轉變,需針對其考慮之運具組合模式進行精準推廣,特別是在通勤高度受限的家庭與女性族群,通勤行為經常是交通決策的核心模式,也是
MaaS 設計與導入時的重要參照。

\textbf{Duan 等人(2022)} 運用類神經網路(Artificial Neural Network,
ANN)模型,針對澳洲地區進行 MaaS
使用意圖之預測,並針對不同出行目的(社交、日常、工作)分別建構模型。結果顯示,通勤目的之
MaaS
使用行為最具預測性,其主要受年齡、就業、子女數、旅行距離與時間、公共運輸票證持有等因素顯著影響。研究指出,通勤者相較於其他出行目的者,具備更穩定的行為模式與明確的需求特徵,為
MaaS
發展初期最具潛力的使用族群之一。此研究亦補足過往文獻對旅次目的差異之探討不足,顯示不同出行情境下的
MaaS 採用傾向需納入政策與系統設計的考量。

綜上所述,從潛在使用者及使用意願到以使用特性及目的進行之行為預測研究,於
MaaS
使用者行為特性之研究:以通勤目的為主並透過多運具組合之公共運輸使用、科技接受度及環保意識較高之使用者,是為MaaS推廣的主要受眾。下節將進一步轉向以實際使用行為紀錄為基礎的研究,探討
MaaS
使用行為的時空特性與群體異質性,彌補僅從意願推論而未驗證實際行為的限制。

\subsection{MaaS使用者行為}\label{maasux4f7fux7528ux8005ux884cux70ba}

於MaaS使用者行為研究,目前多聚焦於以下方向:轉乘行為、公共運輸使用行為。透過轉乘共享運具(機車)使用資料去了解MaaS使用者進行第一及最後一哩路之轉乘行為分析;透過票證資料進行公共運輸使用行為分析。以下為研究細述。

\textbf{盧宗成 等人 (2021)} 以高雄市MaaS系統 - MeNGo
之使用者作為研究對象,分析MaaS系統中公共運輸和共享運具間的轉乘行為。研究透過比對公共運輸進出站點及時間與共享機車之借還車時間及座標,判定該旅次是否為構成公共運輸與共享運具之轉乘旅次。結果顯示,於包含共享運具之旅次中,約有
53\%
為公共運輸與共享運具間轉乘之旅次。其中以捷運系統與共享運具間之轉乘行為比例最高,其次為公車及共享自行車。研究進一步以負二項回歸模型發現,年輕族群,且戶籍地行政區擁有捷運站點之使用者,具有較高強度之公共運輸與共享運具間之轉乘行為,研究建議將此族群作為MaaS服務之主要推廣對象,並指出於MaaS服務中,共享運具能有效作為公共運輸路網中第一及最後一哩路之接駁工具,亦能補足部份路網之公共運輸涵蓋範圍較低的問題。

\textbf{郭 (2019)} 以高雄市 MaaS 系統 - MeNGo
作為研究對象,透過公共運輸電子票證資料進行使用者分群分析,研究透過 14
個使用者特徵變數,其包含使用金額、運輸行為及空間距離,三大使用者行為構面,並結合無監督學習方法
k-means ,將 MeNGo
使用者以學生及一般族群作為大類,向下細分為四類一般族群及五類學生族群,共劃分為九個集群,研究進一步結合使用者分群與國外成功
MaaS 案例,針對高雄市 MaaS 系統 - MeNGo
提出後續營運策略改善建議以及公共運輸服務品質改善之建議。

綜上所述,MaaS 使用者行為研究在共享運具及公共運輸領域上皆已有涉略,郭
(2019)
在研究中提出九類使用者族群,針對高雄市MaaS系統之使用者以金額、旅運行為等構面進行使用者特性分析,是為對整體MaaS使用者行為特性之綜述,研究中缺少針對通勤者之旅次特性進行詳細描述,本研究後續將針對公共運輸使用者行為特徵之研究進行回顧,以了解公共運輸使用者行為特徵,藉此了解應該要選用哪些特徵作為MaaS使用者之公共運輸使用特徵之萃取。

\section{公共運輸電子票證及使用者行為特徵}\label{ux516cux5171ux904bux8f38ux96fbux5b50ux7968ux8b49ux53caux4f7fux7528ux8005ux884cux70baux7279ux5fb5}

本研究藉由回顧先前研究針對公共運輸電子票證資料所萃取之使用者行為特徵進行回顧,透過電子票證資料可以更有效地去確切知道每個使用者的公共運輸使用行程紀錄,相較過去研究使用問卷調查方法進行使用者旅次目的、特性等的推論,透過電子票證去分析可以更好地去捕捉使用者的完整行程及特性,藉此進行更細緻的使用者行為及特性的推斷。以下為本研究回顧之相關研究。

公共運輸使用頻率特性可以去理解使用者對於整體公共運輸系統的依賴程度,了解使用者是否主要透過公共運輸系統進行日常及通勤等運輸行為,並藉此了解運輸系統如何影響使用者的日常出行模式,Ma
等人 (2013)
提出旅次鏈的概念,以北京市公共運輸系統為例,透過轉乘時間門檻及交易行為將使用者每日運輸行為合併為單一旅次鏈,接著透過DBSCAN演算法對相似旅次鏈進行萃取,並以使用天數、相似路線、相似站點與相似出發時間等四項使用者旅次特性作為特徵,以K-means++演算法將使用者分為五類規律程度;Kieu
等人(2015)進一步以使用者出行之固定起訖點與時段,以雙層 DBSCAN
於個體層面進行進出站點進行分群,接著依規則式分群方法針對使用者旅行特性進行分群,辨識出四類各具相異使用頻率及規律性之使用者群體,其中於該研究案例中指出,通勤者定義同時具有固定
起訖點及固定上車時段之使用者,其在整體樣本中僅佔 14\%,但貢獻了 46\%
的票收收入。這類使用者大多於平日時段出現,且 90\%
以上行程皆不需轉乘,顯示其對於出行效率具有較高要求。透過使用頻率、時間及空間等使用特性描述使用者之旅次行為,得以去理解在公共運輸系統中主要依賴的使用者族群。

於使用者出行規律性之研究,近年來漸引入資訊熵作為描述使用者出行特性的重要指標。Yang
等人(2017)藉由聯合熵量化使用者於時空組合上的不確定性,以熵為基礎辨識出三類不同出行規律性之使用者,並以
Markov/HMM 進行驗證,指出熵值較低之使用者可預測性達
84.5\%;Goulet-Langlois
等人(2018)則以熵作為衡量多日旅次序列之特徵,證實可同時辨識穩定通勤者與多目的之使用者;此外,Sun
等人(2024)亦透過資訊熵衡量使用者出行時間與地點的分散程度,提出結合熵值與分群之架構,分別針對通勤與非通勤活動進行分析,並成功辨識出不同特性的通勤族群(如規律上班族、彈性上班族)與非通勤族群(如主動探索者、限制型訪客),證實以資訊熵為基礎的特徵具有良好的解釋力,可有效捕捉出行的規律程度並區別使用者群體間之潛在異質性。綜合以上研究,透過資訊熵對使用者出行特性進行量化及描述,能有效根據出行規律性的差異將使用者予以區辨;特別是低熵值的使用者,因其在出行時間與空間上的高度穩定性,故具有更佳的行為預測與辨識能力,亦即穩定通勤之使用者族群。此亦為本研究以熵作為使用者特性描述之重要依據,用以詳實理解
MaaS 使用者之通勤特性。

於公共運輸使用者的出行行為分析中,時空特徵與先前所述之規律性與資訊熵之結合逐漸成為辨識使用者類型及其出行模式的重要依據。Liu(2019)以台北市公共運輸電子票證資料為基礎,從時間、空間、活動模式、運具使用組合及社經特性等多維度萃取出使用者特徵,並透過
k-prototype 演算法進行分群研究,成功辨識不同使用模式之使用者集群。Lin
等人(2020)則進一步提出了一套結合電子票證資料與旅次目調查資料的兩階段分群與分類模型,以六項特徵指標描述通勤使用者在時間與空間維度的特性,包括出行日數、出行次數、路線使用數量、對稱性旅次日數,以及時間熵與地點熵。研究指出,通勤者於時間與空間熵值明顯較低,對稱性與常用路線比例亦顯著高於其他群體。此外,Pieroni
等人(2021)以巴西聖保羅智慧卡資料為基礎,著重分析低收入區域居民的通勤行為,設計包含時間(如每日首筆交易時間的中位數與變異)、空間(如每日最長行程距離的中位數與變異)、活動(如每日首末筆交易間隔)及社經與土地使用等四大類別共十四項變數,透過
DBSCAN 方法推估居住區位,最後透過 K-means 演算法及 Silhouette
Coefficient 確定適當的分群數,成功識別出不同特性之通勤族群。

綜合上述研究可發現,將時間與空間行為特徵結合資訊熵與分群方法,有助於更精確地捕捉使用者的出行規律性與穩定性,並藉此針對集群特性進行詳細的特性描述,區分出不同的通勤及活動族群。

\section{研究缺口}\label{ux7814ux7a76ux7f3aux53e3}

綜觀上述文獻,現有針對 MaaS
使用者行為的研究多偏重於使用意願、主觀偏好等心理或態度層面的探討,對於基於實際資料之客觀分析仍有明顯不足,尤其是缺乏針對公共運輸系統中通勤行為的細緻分析。儘管部分研究已有使用電子票證資料進行分群的嘗試,但多停留於表層的特徵描述,未有效整合過去研究已證實的重要旅次特性(如對稱性、資訊熵)進行深入且系統性的分析與辨識。具體而言,本研究明確指出以下研究缺口:

現有 MaaS
文獻缺乏透過使用者實際電子票證使用資料,結合多維度的旅次特徵(如旅次對稱性、資訊熵、轉乘行為等),以更全面、深入地探討並分類通勤者的具體行為模式。

藉此,本研究以高雄市 MaaS 系統之電子票證資料為實證基礎,透過 PCA
方法進行特徵降維,並結合 K-means++
演算法進行穩定且精確的使用者分群,以填補上述研究缺口。本研究的結果可提供更具體且實務導向的政策參考,協助改善
MaaS 系統的服務規劃與營運策略。
