% !TeX root = ../main.tex
\chapter{文獻回顧}\label{ux6587ux737bux56deux9867}

本研究針對交通行動服務及公共運輸電子票證資料相關研究進行文獻回顧及評析。

\section{交通行動服務(Mobility as a
Service)}\label{ux4ea4ux901aux884cux52d5ux670dux52d9mobility-as-a-service}

綜觀目前國內研究,MaaS系統相關研究多聚焦於使用者意向的質性分析。多篇研究均指出MaaS是一個整合性的交通服務平台,結合多種交通工具、票務系統、旅行規劃等功能。使用者透過單一平台即可完成所有旅程規劃、訂票和支付(陳,
2021; 何, 2021; 邱, 2020; 周, 2020),以下為目前主要的研究內容及方向。

近年來,台灣對MaaS的研究呈現多元化趨勢,涵蓋觀光應用、城市間比較、服務套裝方案設計、使用者行為分析等多個面向。這些研究普遍採用敘述性偏好法收集數據,並運用各種統計模型如羅吉特模型、潛在類別模型、個體選擇模型等進行分析。(陳,
2021)和(張,
2023)的研究分別探討了MaaS在觀光領域的應用潛力和不同城市環境下的實施差異,為MaaS的場景化應用提供重要參考。

(邱, 2020)和(何,
2021)的研究中主要針對MaaS服務方案的設計和定價策略,通過條件評估法和各種選擇模型,識別了不同用戶群體的偏好,並提出了針對性的方案設計建議。這些研究不僅計算了用戶對不同交通服務的支付意願,亦識別出了如綠色運輸使用者、小客車使用者等不同類型的MaaS使用者群體,為服務提供商制定差異化策略提供了依據。

(周,
2020)的研究則採用了決策樹模型,以實際運營的高雄市MaaS系統為例,分析用戶的套票購買行為。這項研究建立了會員方案續買的預測模型,並識別出了影響續買行為的關鍵因素,同時提出了處理資料不平衡問題的創新方法。

(郭,
2019)針對高雄MeNGo交通行動服務(MaaS)定期票顧客的使用行為進行了深入研究。該研究通過建立電子票證資料處理流程,並使用K-Means集群分析方法,將MeNGo無限暢遊方案顧客分為九種不同類型。

這些研究成果為台灣MaaS系統的實際運營及後續發展提供了建議,有助於推動MaaS在台灣的進一步發展和應用。

於國外MaaS相關的研究中,主要也以問卷方式進行分析為主,

\subsection{交通行動服務
文獻小結}\label{ux4ea4ux901aux884cux52d5ux670dux52d9-ux6587ux737bux5c0fux7d50}

\section{公共運輸電子票證}\label{ux516cux5171ux904bux8f38ux96fbux5b50ux7968ux8b49}

在以電子票證進行使用者分群之研究中,國內多數以現有公共運輸系統為主,尚未有將MaaS系統和公共運輸系統相互整合並提出痛點之研究。以下為結合公共運輸電子票證在使用者分群之常見方法:

\subsection{公共運輸電子票證
文獻小結}\label{ux516cux5171ux904bux8f38ux96fbux5b50ux7968ux8b49-ux6587ux737bux5c0fux7d50}

\section{文獻回顧小結}\label{ux6587ux737bux56deux9867ux5c0fux7d50}

MaaS本身是為一種面向使用者本身之服務,多數研究皆以質性研究為主,是為探討使用者之意向。目前國內較多以質性方法進行MaaS相關的研究。無論國內外,目前皆較少同時針對MaaS及公共運輸的使用者並以票證資料進行量化分析的研究,並在國內外的研究中都顯示,MaaS系統可以有效去提高城市中的公共運輸使用,。