% !TeX root = ../main.tex
\chapter{緒論}\label{ux7dd2ux8ad6}

\section{研究背景}\label{ux7814ux7a76ux80ccux666f}

本研究針對高雄市現行之交通行動服務(Mobility as a Service,
MaaS)系統MeNGo為研究對象,透過MaaS系統使用者票證資料萃取特徵,從時間序列、出行穩定度等面向進行特徵萃取,其目的為了解現行在MaaS系統使用者之旅行行為,並建立各使用者群體之旅運行為模型,以找出目前MaaS服務尚未涵蓋之使用者群體。以下為本研究之進度說明。

交通行動服務(MaaS)概念最早於2014年在芬蘭提出。2021年,世界經濟論壇(WEF)將MaaS定義為一種整合並無縫提供多種既有移動運輸服務的系統。MaaS系統以個人行動裝置為主要平台,通過應用程式為基礎,提供使用者依其旅運需求選擇平台所提供之行程套裝服務、運具規劃等功能。MaaS的主要目標是讓使用者透過個人行動裝置完成所有運輸需求及公共運輸相關服務,並以減少私人運具使用為願景,為城市及社會提供永續、便捷且安全的運輸環境。

2018年起,MeNGo成為亞洲首個MaaS示範區,整合高雄市轄內各項公共運輸運具至MeNGo系統中,陸續發行月票、週票及小時票,以服務具有各項公共運輸需求之民眾。此外,MeNGo亦將共享電動運具、停車場及計程車轉乘優惠點數等誘因導入系統中。

\section{研究動機}\label{ux7814ux7a76ux52d5ux6a5f}

因應行政院之Tpass政策,現今國內公共運輸票價獲得較以往更多的補貼。MeNGo原先在高雄地區提供的全運具使用套票方案,其價格從1,499元/月降低至399元/月,於先前有關高雄市公共運輸定期票使用者的研究中將高雄市公共運輸使用者劃分為七個集群(郭,
2019),然而該研究年分與現今相近,但在公共運輸定期票價格及高雄市公共運輸路網上已有顯著差異。定期票價格下降和環狀輕軌成環後,可能導致公共運輸使用者群集產生變化。
此外,鑑於第一期政策年期將近(2025年),未來若無繼續對MeNGo套票進行補助,如何持續維持現有使用者或開發潛在使用者成為重要課題

\section{研究目的}\label{ux7814ux7a76ux76eeux7684}

\section{研究架構}\label{ux7814ux7a76ux67b6ux69cb}